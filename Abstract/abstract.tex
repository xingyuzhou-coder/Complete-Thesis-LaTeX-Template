% ************************** Thesis Abstract *****************************
% Use `abstract' as an option in the document class to print only the titlepage and the abstract.

\begin{abstract}

Microelectromechanical system (MEMS) is a multidisciplinary and cutting-edge scientific research field produced and developed on the basis of microelectronics technology. It is a micro device or system that integrates micro sensors, micro actuators, and micro mechanical structures. Traditional MEMS devices are usually based on silicon materials, which are showing more and more environmental limitations and temperature reliability problems. These shortcomings of silicon-based MEMS technology have promoted the research of more biochemically resistant and thermally stable MEMS such as wide-bandgap group III-V nitride semiconductor material MEMS. Group III-V nitride materials have very high mechanical, thermal and chemical stability, as well as excellent high-frequency characteristics. Its high-concentration two-dimensional electron gas (2DEG) is extremely sensitive to mechanical loading and surface chemical modification. In addition, group III-V nitride materials have unique piezotronics effect. These characteristics greatly expand the application of III-V nitride materials in the MEMS field. Based on the piezoelectric effect, this thesis systematically studies the theoretical modeling and fabrication of gallium nitride power MEMS devices based on the AlGaN/AlN/GaN heterojunction with microcantilever structure. A semi-classical physical model of the power MEMS is established, and a strain-regulated power MEMS device (Strain-controlled Power Device, SPD) and a magnetic field-controlled power MEMS device (Magnetosensory Power Device, MPD) are also prepared, which provides a theoretical framework and novel device structure for the research of III-V nitride power MEMS devices. 

\noindent The content of this thesis is mainly composed of the following three parts:\\

\noindent 1. Theoretical model of power MEMS devices \\

\noindent Based on the theory of piezoelectric effect and semiconductor physics, a semi-classical physical model of a MEMS cantilever device based on the AlGaN/AlN/GaN heterojunction is established, and the modulation characteristics of the external stress on the energy band of the heterojunction as well as the electrical performance of the MEMS device are calculated, which provides a theoretical framework for the development of SPD and MPD, and the research guidance for the development and optimization of cantilevered III-V nitride power MEMS devices.\\

\noindent 2. Manufacturing of power MEMS devices \\

\noindent In this study, the manufacture of GaN power MEMS devices from epitaxial growth wafers to well-functional devices have be systematically studied. Benefiting from the rapid development of III-V compound semiconductor fabrication and characterization equipment. Firstly, the main nanofabrication and characterization equipment is introduced, including epitaxial growth, dry etching, photolithography, thin film deposition, plasma cleaning, Raman spectroscopy, scanning electron microscopy, transmission electron microscopy, etc. Secondly, the corresponding process parameters and their process integration have been developed based on the manufacturing technology and equipment, thus realizing the whole process from GaN wafer to device. Step by step, high-performance GaN HEMTs and GaN power MEMS devices have been successfully fabricated.\\

\noindent 3. Strain-controlled / Magnetosensory power MEMS devices\\

\noindent In this study, a strain-modulated power MEMS device (Strain-controlled power device, SPD) and a magnetic field-controlled power MEMS device (Magnetosensory power devices, MPD) are designed. The SPD uses external strain to modulate the output power of the device by simulating the reflection process of the human body. Under the external strain of 0 $\sim$ 16 \unit{mN}, the maximum output power density of SPD increases from \num{2.30e3} \unit{\watt\per\square\cm} to \num{2.72e3} \unit{\watt\per\square\cm}. The MPD uses an external magnetic field to modulate the output power of the device by simulating the working mechanism of the magnetic induction neuron. Upon the magnetic field of 200 \unit{\milli\tesla}, the maximum output power density of the MPD reached 85.8 \unit{\watt\per\square\mm}. Under the action of an external magnetic field of 0 $\sim$ 400 \unit{\milli\tesla}, when the gate voltage is -5 \unit{\volt}, the saturation output power density of MPD increases from 18.04 \unit{\watt\per\square\mm} to 18.94 \unit{\watt\per\square\mm}. Both the SPD's and MPD's gate voltage can control the output power in a larger range, so it combines the two-dimensional control advantages of small-range external strain control and large-range programmable gate voltage control. This research not only provides insights into the interaction between external stimulation and power control of GaN MEMS devices, but also promotes the development of bionic intelligent power MEMS devices.\\

\end{abstract}